% arara: pdflatex
\documentclass[a4paper, ngerman, landscape, fleqn]{article}

\usepackage[utf8]{inputenc}
\usepackage[margin=0.5cm, includefoot]{geometry}
\usepackage{multicol}
\usepackage{graphicx}
\usepackage[hidelinks]{hyperref}
\usepackage{fancyhdr}
\usepackage{lastpage}
\usepackage[ngerman]{babel}
\usepackage{titlesec}
\usepackage{calc}
\usepackage{enumitem}
\usepackage{amsmath}
\usepackage{amssymb}
\usepackage{ifthen}
\usepackage{color}
\usepackage{textcase}
\usepackage{capt-of}
\usepackage{listings}
\usepackage{stmaryrd}
\usepackage{mathrsfs}
\usepackage{algpseudocode}
\usepackage{tikz}
\usepackage{mathtools}

\usepackage{lipsum}	% für Dummy-Text

%
\hypersetup{
	unicode=true,
	pdftitle={Mathematik für die Datenkommunikation},
	pdfauthor={Mathias Graf},
	pdfsubject={Zusammenfassung Mathematik für die Datenkommunikation},
	pdfcreator={LaTeX},
	pdfproducer={pdflatex},
	pdfnewwindow=true
}

% Abstand und Linienbreite zwischen Spalten
\setlength{\columnsep}{1cm}
\setlength{\columnseprule}{0.4pt}

% kein Einrücken bei neuem Absatz
\setlength{\parindent}{0pt}

% Weniger Abstand vor und nach überschriften
\titlespacing\section{0pt}{12pt plus 4pt minus 2pt}{0pt plus 2pt minus 2pt}
\titlespacing\subsection{0pt}{12pt plus 4pt minus 2pt}{0pt plus 2pt minus 2pt}
\titlespacing\subsubsection{0pt}{12pt plus 4pt minus 2pt}{0pt plus 2pt minus 2pt}


% Datastructure picture base size
\newlength{\basesize}
\setlength{\basesize}{0.7cm}

% make comments in pseudocode c-style
\renewcommand{\algorithmiccomment}[1]{\textcolor{gray}{// #1}}

\newcommand*\circled[1]{\tikz[baseline=(char.base)]{
		\node[shape=circle,draw,inner sep=1pt] (char) {#1};}}

\pagestyle{fancy}

% Footer konfigurieren
\makeatletter
\lfoot{\@title}
\cfoot{\today}
\rfoot{Seite~\thepage~von~\pageref{LastPage}}
\makeatother
\renewcommand{\footrulewidth}{0.4pt}							% hrule über footer

\renewcommand{\labelitemi}{$\vcenter{\hbox{\tiny$\bullet$}}$}	% kleinere Dots in Itemize

\newcommand{\Mod}{\ \mathrm{mod}\ }

% Titel und Author
\title{Zusammenfassung Mathematik für die Datenkommunikation}
\author{Mathias Graf}
\def\email{\href{mailto:mathias.graf@students.fhnw.ch}{\texttt{mathias.graf@students.fhnw.ch}}}

\begin{document}

% Titel setzen
\makeatletter
{\Large \textbf{\@title}}
\hfill
{\@author}
\hfill
\email
\makeatother
\hrule

\begin{multicols*}{3}

\section*{Teilbarkeit}
Seien $a, b, c, d \in \mathbb{Z}$
\begin{align*}
    a \backslash a \\
    a \backslash b \wedge b \backslash a \implies a = b \vee a = -b \\
    a, b \geq 0 \wedge a \backslash b \wedge b \backslash a \implies a = b \\
    a \backslash b \wedge b \backslash c \implies a \backslash c \\
    a \geq 0 \wedge b > 0 \wedge a \backslash b \implies a \leq b \\
    1 \backslash a \wedge -1 \backslash a \wedge a \backslash a \wedge -a \backslash a \\
    a \backslash 1 \vee a \backslash -1 \implies a = 1 \vee a = -1 \\
    a \backslash 0 \\
    0 \backslash a \implies a = 0 \\
    a \backslash b \implies a \backslash -b \wedge -a \backslash b \wedge -a \backslash -b \\
    a \backslash b \implies (\forall x \in \mathbb{Z} | a \backslash (b \cdot x)) \\
    a \backslash b \implies (\forall x \in \mathbb{Z} | (a \cdot x) \backslash (b \cdot x)) \\
    (a \cdot b) \backslash c \implies a \backslash c  \wedge b \backslash c \\
    a \backslash b \wedge c \backslash d \implies (a \cdot c) \backslash (b \cdot d) \\
    a \backslash b \wedge a \backslash c \implies (\forall u, v \in \mathbb{Z} | a \backslash (u \cdot b + v \cdot c))
\end{align*}

\section*{Euklidsches Divisionslemma}
Seien $a, b \in \mathbb{Z}, b \neq 0$. Dann gibt es eindeutige $q, r \in \mathbb{Z}$ mit:
\begin{equation*}
    a = b \cdot q + r \wedge 0 \leq r < | b | 
\end{equation*}

\section*{Normen der ganzzahligen Division}
Seien $a, b \in \mathbb{Z}$ mit $b \neq 0$. Dann gibt es eindeutige $q_e, r_e, q_f, r_f, q_t r_t \in \mathbb{Z}$ mit

\begin{align*}
    a = b \cdot q_e + r_e &\wedge 0 \leq r_e < | b | \\
    a = b \cdot q_f + r_f &\wedge 
    \begin{cases*}
        0 \leq r_f < b & if $b > 0$ \\
        b < r_f \leq 0 & if $b < 0$
    \end{cases*} \\
    a = b \cdot q_t + r_t &\wedge 
    \begin{cases*}
        0 \leq r_t < |b| & if $a \geq 0$ \\
        -|b| < r_t \leq 0 & if $a \leq 0$
    \end{cases*}
\end{align*}


\section*{Lemma von Bézout}
\begin{equation*}
    \forall a, b \in \mathbb{Z} \exists s, t \in \mathbb{Z} : gcd(a, b) = s \cdot a + t \cdot b
\end{equation*}

\section*{GCD Definition}
\begin{align*}
    C_{a, b}, L_{a, b}, D_{a, b} : \mathbb{Z} \xrightarrow{total} \mathbb{B} \\
    C_{a, b}(d) := d \backslash a \wedge d \backslash b \\
    L_{a, b}(d) := (\exists u, v \in \mathbb{Z} | u \cdot a + v \cdot b = d) \\
    D_{a, b}(d) := (\forall c \in \mathbb{Z} | c \backslash a \wedge c \backslash b \implies c \backslash d)
\end{align*}

\section*{Fundamentale Eigenschaften GCD}
Seien $a, b, c, k \in \mathbb{Z}$. Dann gilt:

\begin{align*}
    \gcd(a, b) = 0 \iff a = 0 \wedge b = 0 \\
    \gcd(a, 0) = |a| \\
    \gcd(a, 1) = 1 \\
    \gcd(a, a) = |a| \\
    \gcd(a, b) = \gcd(b, a) \\
    \gcd(a, b) = \gcd(|a|, |b|) \\
    \gcd(a, b) = \gcd(a, b + k \cdot a) \\
    \gcd(a, b) = \gcd(a, b - a) \\
    a \neq 0 \implies \gcd(a, b) = \gcd(a, b \Mod a) \\
    \gcd(a, \gcd(b, c)) = \gcd(\gcd(a, b), c) \\
    \gcd(c \cdot a, c \cdot b) = |c| \cdot \gcd(a, b) \\
    a \neq 0 \wedge b \neq 0 \implies \gcd(a, b) \leq min(|a|, |b|)
\end{align*}

\section*{Erw. Euklidscher Algorithmus}

Der erweiterte euklidsche Algorithmus berechnet zusätzlich zum GCD auch die Bézout-Koeffizienten.

\begin{algorithmic}
    \Function{extended\_gca}{a, b}
        \State (old\_r, r) := (a, b)
        \State (old\_s, s) := (1, 0)
        \State (old\_t, t) := (0, 1)

        \While{$r \neq 0$}
            \State quotient := old\_r \textbf{div} r
            \State (old\_r, r) := (r, old\_r - quotient * r)
            \State (old\_s, s) := (s, old\_s - quotient * s)
            \State (old\_t, t) := (t, old\_t - quotient * t)
        \EndWhile

        \State \textbf{output} "Bézout-Koeffizienten:", (old\_s, old\_t)
        \State \textbf{output} "Greatest Common Divisor:", old\_r
        \State \textbf{output} "Quotients by the GCD:", (t, s)

    \EndFunction
\end{algorithmic}

\section*{Abstrakte Algebra}

\subsection*{Halbgruppe}
Eine Halbgruppe $S = (S, \circ)$ besteht aus einer Menge $S$ und einer inneren zweistelligen Verknüpfung 
\begin{equation*}
    \circ : S \times S \rightarrow S, (a, b) \mapsto a \circ b
\end{equation*}

die assoziativ ist, d.\,h.
\begin{equation*}
    (\forall a, b, c \in S | a \circ (b \circ c) = (a \circ b) \circ c)
\end{equation*}

\subsection*{Gruppe}
Eine Gruppe $S = (S, \circ)$ besteht aus einer Menge $S$ und einer inneren zweistelligen Verknüpfung
\begin{equation*}
    \circ : S \times S \rightarrow S, (a, b) \mapsto a \circ b
\end{equation*}

die folgenden Gruppenaxiome gennanten Forderungen:
\begin{align*}
    (a \circ b) \circ c = a \circ (b \circ c) \\
    (\exists e \in G : \forall a \in G a \circ e = e \circ a = a) \\
    (\forall a \in G : \exists a^{-1} : a \circ a^{-1} = a^{-1} \circ a = e)
\end{align*}
genügen.

\subsection*{Kommutativität}
Eine Halbgruppe, Gruppe oder ein Monoid heissen ``kommutativ'' oder ``abelsch'' wenn für alle $a, b \in S$ gilt:
\begin{equation*}
    b \circ a = a \circ b
\end{equation*}


\end{multicols*}

\end{document}


