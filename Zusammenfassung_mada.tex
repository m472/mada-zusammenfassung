% arara: pdflatex
\documentclass[a4paper, ngerman, landscape, fleqn]{article}

\usepackage[utf8]{inputenc}
\usepackage[margin=0.5cm, includefoot]{geometry}
\usepackage{multicol}
\usepackage{graphicx}
\usepackage[hidelinks]{hyperref}
\usepackage{fancyhdr}
\usepackage{lastpage}
\usepackage[ngerman]{babel}
\usepackage{titlesec}
\usepackage{calc}
\usepackage{enumitem}
\usepackage{amsmath}
\usepackage{amssymb}
\usepackage{ifthen}
\usepackage{color}
\usepackage{textcase}
\usepackage{capt-of}
\usepackage{listings}
\usepackage{stmaryrd}
\usepackage{mathrsfs}
\usepackage{algpseudocode}
\usepackage{tikz}
\usepackage{mathtools}
\usepackage{tabularx}
\usepackage{stackengine}

\usepackage{lipsum}	% für Dummy-Text

%
\hypersetup{
	unicode=true,
	pdftitle={Mathematik für die Datenkommunikation},
	pdfauthor={Mathias Graf},
	pdfsubject={Zusammenfassung Mathematik für die Datenkommunikation},
	pdfcreator={LaTeX},
	pdfproducer={pdflatex},
	pdfnewwindow=true
}

% Abstand und Linienbreite zwischen Spalten
\setlength{\columnsep}{1cm}
\setlength{\columnseprule}{0.4pt}

% kein Einrücken bei neuem Absatz
\setlength{\parindent}{0pt}

% Weniger Abstand vor und nach überschriften
\titlespacing\section{0pt}{12pt plus 4pt minus 2pt}{0pt plus 2pt minus 2pt}
\titlespacing\subsection{0pt}{12pt plus 4pt minus 2pt}{0pt plus 2pt minus 2pt}
\titlespacing\subsubsection{0pt}{12pt plus 4pt minus 2pt}{0pt plus 2pt minus 2pt}


% Datastructure picture base size
\newlength{\basesize}
\setlength{\basesize}{0.7cm}

% make comments in pseudocode c-style
\renewcommand{\algorithmiccomment}[1]{\textcolor{gray}{// #1}}

\newcommand*\circled[1]{\tikz[baseline=(char.base)]{
		\node[shape=circle,draw,inner sep=1pt] (char) {#1};}}

\pagestyle{fancy}

% Footer konfigurieren
\makeatletter
\lfoot{\@title}
\cfoot{\today}
\rfoot{Seite~\thepage~von~\pageref{LastPage}}
\makeatother
\renewcommand{\footrulewidth}{0.4pt}							% hrule über footer

\renewcommand{\labelitemi}{$\vcenter{\hbox{\tiny$\bullet$}}$}	% kleinere Dots in Itemize

\newcommand{\Mod}{\ \mathrm{mod}\ }
\newcommand{\plusdot}{\stackanchor[1pt]{$+$}{$\boldsymbol{\cdot}$}}
\newcommand{\plusdotm}{\stackanchor[1pt]{$+_m$}{$\boldsymbol{\cdot}_m$}}

\newcolumntype{Y}{>{\centering\arraybackslash}X}

% Titel und Author
\title{Zusammenfassung Mathematik für die Datenkommunikation}
\author{Mathias Graf}
\def\email{\href{mailto:mathias.graf@students.fhnw.ch}{\texttt{mathias.graf@students.fhnw.ch}}}

\begin{document}

% Titel setzen
\makeatletter
{\Large \textbf{\@title}}
\hfill
{\@author}
\hfill
\email
\makeatother
\hrule

\begin{multicols*}{3}

\section*{Teilbarkeit}
Seien $a, b, c, d \in \mathbb{Z}$
\begin{align*}
    a \backslash a \\
    a \backslash b \wedge b \backslash a \implies a = b \vee a = -b \\
    a, b \geq 0 \wedge a \backslash b \wedge b \backslash a \implies a = b \\
    a \backslash b \wedge b \backslash c \implies a \backslash c \\
    a \geq 0 \wedge b > 0 \wedge a \backslash b \implies a \leq b \\
    1 \backslash a \wedge -1 \backslash a \wedge a \backslash a \wedge -a \backslash a \\
    a \backslash 1 \vee a \backslash -1 \implies a = 1 \vee a = -1 \\
    a \backslash 0 \\
    0 \backslash a \implies a = 0 \\
    a \backslash b \implies a \backslash -b \wedge -a \backslash b \wedge -a \backslash -b \\
    a \backslash b \implies (\forall x \in \mathbb{Z} | a \backslash (b \cdot x)) \\
    a \backslash b \implies (\forall x \in \mathbb{Z} | (a \cdot x) \backslash (b \cdot x)) \\
    (a \cdot b) \backslash c \implies a \backslash c  \wedge b \backslash c \\
    a \backslash b \wedge c \backslash d \implies (a \cdot c) \backslash (b \cdot d) \\
    a \backslash b \wedge a \backslash c \implies (\forall u, v \in \mathbb{Z} | a \backslash (u \cdot b + v \cdot c))
\end{align*}

\section*{Euklidsches Divisionslemma}
Seien $a, b \in \mathbb{Z}, b \neq 0$. Dann gibt es eindeutige $q, r \in \mathbb{Z}$ mit:
\begin{equation*}
    a = b \cdot q + r \wedge 0 \leq r < | b | 
\end{equation*}

\section*{Normen der ganzzahligen Division}
Seien $a, b \in \mathbb{Z}$ mit $b \neq 0$. Dann gibt es eindeutige $q_e, r_e, q_f, r_f, q_t r_t \in \mathbb{Z}$ mit

\begin{align*}
    a = b \cdot q_e + r_e &\wedge 0 \leq r_e < | b | \\
    a = b \cdot q_f + r_f &\wedge 
    \begin{cases*}
        0 \leq r_f < b & if $b > 0$ \\
        b < r_f \leq 0 & if $b < 0$
    \end{cases*} \\
    a = b \cdot q_t + r_t &\wedge 
    \begin{cases*}
        0 \leq r_t < |b| & if $a \geq 0$ \\
        -|b| < r_t \leq 0 & if $a \leq 0$
    \end{cases*}
\end{align*}


\section*{Lemma von Bézout}
\begin{equation*}
    \forall a, b \in \mathbb{Z} \exists s, t \in \mathbb{Z} : gcd(a, b) = s \cdot a + t \cdot b
\end{equation*}

\section*{GCD Definition}
\begin{align*}
    C_{a, b}, L_{a, b}, D_{a, b} : \mathbb{Z} \xrightarrow{total} \mathbb{B} \\
    C_{a, b}(d) := d \backslash a \wedge d \backslash b \\
    L_{a, b}(d) := (\exists u, v \in \mathbb{Z} | u \cdot a + v \cdot b = d) \\
    D_{a, b}(d) := (\forall c \in \mathbb{Z} | c \backslash a \wedge c \backslash b \implies c \backslash d)
\end{align*}

\section*{Fundamentale Eigenschaften GCD}
Seien $a, b, c, k \in \mathbb{Z}$. Dann gilt:

\begin{align*}
    \gcd(a, b) = 0 \iff a = 0 \wedge b = 0 \\
    \gcd(a, 0) = |a| \\
    \gcd(a, 1) = 1 \\
    \gcd(a, a) = |a| \\
    \gcd(a, b) = \gcd(b, a) \\
    \gcd(a, b) = \gcd(|a|, |b|) \\
    \gcd(a, b) = \gcd(a, b + k \cdot a) \\
    \gcd(a, b) = \gcd(a, b - a) \\
    a \neq 0 \implies \gcd(a, b) = \gcd(a, b \Mod a) \\
    \gcd(a, \gcd(b, c)) = \gcd(\gcd(a, b), c) \\
    \gcd(c \cdot a, c \cdot b) = |c| \cdot \gcd(a, b) \\
    a \neq 0 \wedge b \neq 0 \implies \gcd(a, b) \leq min(|a|, |b|)
\end{align*}

\section*{Erw. Euklidscher Algorithmus}

Der erweiterte euklidsche Algorithmus berechnet zusätzlich zum GCD auch die Bézout-Koeffizienten.

\begin{algorithmic}
    \Function{extended\_gca}{a, b}
        \State \Comment{Requires:}
        \State \Comment{$a \geq 0 \wedge b \geq 0$}
        \Statex
        \State \Comment{Ensures:}
        \State \Comment{$g = \gcd(a, b) \geq 0$}
        \State \Comment{$|sign| = 1$}
        \State \Comment{Probe 1: $(-sign \cdot u') \cdot g = b$}
        \State \Comment{Probe 2: $(+sign \cdot v') \cdot g = a$}
        \State \Comment{Probe 3: $u \cdot a + v \cdot b = g$}
        \State \Comment{Probe 4: $u' \cdot a + v' \cdot b = 0$}
        \State \Comment{Probe 5: $u \cdot v' - u' \cdot v = sign$}
        \Statex
        \State (g, g$'$) := (a, b)
        \State (u, u$'$) := (1, 0)
        \State (v, v$'$) := (0, 1)
        \State sign := +1
        \Statex
        \While{$g' > 0$}
            \State \Comment Invariante: $0 \leq g \wedge 0 \leq g'$
            \State q := g \textbf{div} g$'$
            \State (g, g$'$) := (g$'$, g - q $\cdot$ g$'$)
            \State (u, u$'$) := (u$'$, u - q $\cdot$ u$'$)
            \State (v, v$'$) := (v$'$, v - q $\cdot$ v$'$) 
            \State sign := -sign
        \EndWhile

        \State \textbf{output} ``Bézout-Koeffizienten:'', (u, v)
        \State \textbf{output} ``Greatest Common Divisor:'', g
        \State \textbf{output} ``Quotients by the GCD:'', (u$'$, v$'$)
    \EndFunction
\end{algorithmic}

\subsection*{Beispiel}
a = 99, b = 78
\vspace{1ex}

\begin{tabularx}{\linewidth}{XXXXXXXX}
    \hline
     g & g$'$ & q &   u & u$'$ &  v & v$'$ & sign \\
    \hline
    99 &   78 &   &   1 &    0 &  0 &    1 &    + \\
       &      & 1 &     &      &    &      &      \\
    78 &   21 &   &   0 &    1 &  1 &   -1 &   $-$\\
       &      & 3 &     &      &    &      &      \\
    21 &   15 &   &   1 &   -3 & -1 &    4 &    + \\
       &      & 1 &     &      &    &      &      \\
    15 &    6 &   &  -3 &    4 &  4 &   -5 &   $-$\\
       &      & 2 &     &      &    &      &      \\
     6 &    3 &   &   4 &  -11 & -5 &   14 &    + \\
       &      & 2 &     &      &    &      &      \\
     3 &    0 &   & -11 &   26 & 14 &  -33 &   $-$\\
    \hline
\end{tabularx}

\section*{Abstrakte Algebra}

\subsection*{Halbgruppe}
Eine Halbgruppe $S = (S, \circ)$ besteht aus einer Menge $S$ und einer inneren zweistelligen Verknüpfung 
\begin{equation*}
    \circ : S \times S \rightarrow S, (a, b) \mapsto a \circ b
\end{equation*}

die assoziativ ist, d.\,h.
\begin{equation*}
    (\forall a, b, c \in S | a \circ (b \circ c) = (a \circ b) \circ c)
\end{equation*}

\subsection*{Monoid}

Ein Monoid $(M, \circ, e)$ ist eine Halbgruppe mit neutralem Element, wobei $\circ$ eine innere zweistellige Verknüpfung ist:
\begin{equation*}
    \circ : M \times M \rightarrow M, (a, b) \mapsto a \circ b
\end{equation*}

\begin{equation*}
    \forall a, b, c \in M : (a \circ b) \circ c = a \circ (b \circ c)
\end{equation*}

\begin{equation*}
    (\exists e \in M : (\forall a \in M : e \circ a = a \circ e = a))
\end{equation*}

Jede Gruppe ist ein Monoid, aber ein Monoid besitzt im Gegensatz zu einer Gruppe nicht notwendigerweise ein inverses Element zu jedem Element.

\subsection*{Gruppe}
Eine Gruppe $S = (S, \circ)$ besteht aus einer Menge $S$ und einer inneren zweistelligen Verknüpfung
\begin{equation*}
    \circ : S \times S \rightarrow S, (a, b) \mapsto a \circ b
\end{equation*}

die folgenden Gruppenaxiome gennanten Forderungen:
\begin{align*}
    (a \circ b) \circ c = a \circ (b \circ c) \\
    (\exists e \in G : \forall a \in G a \circ e = e \circ a = a) \\
    (\forall a \in G : \exists a^{-1} : a \circ a^{-1} = a^{-1} \circ a = e)
\end{align*}
genügen.

\subsection*{Untergruppen, zyklische Gruppen}
Definition: Sei $(M, \circ_1, \dotsc, \circ_m)$ eine Algebra. Sei $N \subseteq M$, sei $N$ abgeschlossen unter $\circ_1, \dotsc, \circ_m$. Dann ist $(N, \circ_1, \dotsc, \circ_m)$ eine Algebra und heisst \emph{Subalgebra} von $(M, \circ_1, \dotsc, \circ_m)$.

Beispiel: $(\mathbb{Z}^*_m, \cdot_m)$ eine Subalgebra von $(\mathbb{Z}_m, \cdot_m)$.

Definition: Sei $G$ eine Gruppe. Eine Subalgebra $S$ von $G$ heisst \emph{Untergruppe (UG)} von $G$, wenn $S$ eine Gruppe ist.

Beispiel: $(2\mathbb{Z}, +, 0, -\bullet)$ ist UG von $(\mathbb{Z}, +, 0, -\bullet)$.

Satz: Sei $G = (M, \circ, 1, \bullet^{-1})$ Gruppe. Sei $a \in M$.
Definiere:
\begin{align*}
    \left< a \right> :&= \{ a^i | i \in \mathbb{Z} \} \\
                      &= \{ \dotsc, a^{-3}, a^{-2}, a^{-1}, a^{0}, a^1, a^2, a^3, \dotsc \}
\end{align*}

Dann ist $S = (\left< a \right>, \circ, 1, \bullet^{-1})$ UG von $G$ mit $1 = a^0$ neutral und $a^{-n}$ invers zu $a^n$.

Definition: $(\left<a\right>, \circ)$ heisst die \emph{von $a$ erzeugte UG} von $G$. $a$ heisst \emph{Generator} der UG.

Definition: Sei $G = (M, \circ)$ Gruppe. Falls $\exists a \in M : \left<a\right> = M$, dann heisst $G$ \emph{zyklische Gruppe}, $a$ heisst \emph{Generator} von $G$.

Beispiel: $(\mathbb{Z}_6, +_6, 0, -_6\bullet)$
\begin{align*}
    \left<2\right> &= \{ i \times 2 | i \in \mathbb{Z} \} \\
                   & \begin{multlined}
                       = \{ \dotsc, (-3) \times 2, (-2) \times 2, (-1) \times 2, \\
                           0 \times 2, 1 \times 2, 2 \times 2, 3 \times 2, \dotsc \}
                     \end{multlined} \\
                   & \begin{multlined}
                       = \{ \dotsc, 3 \times (-_6 2), 2 \times (-_6 2), 1 \times (-_6 2), \\
                       0 \times 2, 1 \times 2, 2 \times 2, 3 \times 2, \dotsc \}
                     \end{multlined} \\
                   &= \{ \dotsc, 0, 2, 4, 0, 2, 4, 0, \dotsc \} \\
                   &= \{ 0, 2, 4 \}\ \textcolor{gray}{\in \mathbb{Z}_6 = \{ 0, 1, 2, 3, 4, 5 \}}
\end{align*}

\begin{align*}
    \left< 0 \right> &= \{ 0 \} \\
    \left< 1 \right> &= \left< 5 \right> = \mathbb{Z}_6 \\
    \left< 2 \right> &= \left< 4 \right> \\
    \left< 3 \right> &= \{ 0, 3 \}
\end{align*}

Somit ist $\mathbb{Z}_6$ \emph{zyklisch} da $\left< 1 \right> = \left< 5 \right> = \mathbb{Z}_6$.

Definition: $G = (M, \circ, 1, \bullet^{-1})$ Gruppe. $a \in M$.
Falls $m \in \mathbb{Z}^{\neq 0}$ existiert mit $a^m = 1$ dann existiert eine \emph{kleinste positive Zahl} $n \in \mathbb{N}^+$ mit $a^n = 1$. 

Diese Zahl $n$ heisst \emph{Ordnung} von $a$, $n= \mathrm{ord}(a)$.

Beispiel: $(\mathbb{Z}_6, +_6, 0, -_6\bullet); \mathrm{ord}(2)=\ ?$
\begin{align*}
    1 \times 2 &= 2 \neq 0 \\
    2 \times 2 &= 2 +_6 2 = 4 \neq 0 \\
    3 \times 2 &= 2 +_6 2 +_6 2 = 0 \\
               &\implies \mathrm{ord}(2) = 3
\end{align*}
weil $3 \times 2 = 0$ \underline{und} $2 \times 2 \neq 0$ \underline{und} $1 \times 2 \neq 0$.

\subsection*{Kommutativität}
Eine Halbgruppe, Gruppe oder ein Monoid heissen ``kommutativ'' oder ``abelsch'' wenn für alle $a, b \in S$ gilt:
\begin{equation*}
    b \circ a = a \circ b
\end{equation*}

\section*{Diophantische Gleichungen}
\subsection*{Kongruenz}
Seien $a, b, m \in \mathbb{Z}$. Dann heissen $a$ und $b$ \emph{kongruent modulo} $m$ wenn
\begin{equation*}
    m \backslash (a -b)
\end{equation*}
gilt.

Dies wird als 
\begin{equation*}
    a \equiv_m b \ \ \text{oder}\ \ a \equiv b (\Mod m)
\end{equation*}
geschrieben.

\vspace{1em}

Das heisst:
\begin{equation*}
    \exists k \in \mathbb{Z} : a = b + k \cdot m
\end{equation*}

\subsection*{Menge der Reste modulo m}

Sei $m \in \mathbb{N}^+$. Dann heisst
\begin{equation*}
    \mathbb{Z}_m := \{0,1,\dots, m - 1\}
\end{equation*}
Menge der Reste modulo m.

\subsection*{Kongruenzsatz 1}
Seien $a, b \in \mathbb{Z}, m \in \mathbb{N}^+$. Dann gilt:
\begin{align*}
    a \equiv_m b & \iff a \Mod m = b \Mod m \\
    a \in \mathbb{Z}_m & \iff a \Mod m = a
\end{align*}

\subsection*{Kongruenzsatz 2}
Seien $a, b \in \mathbb{Z}; m \in \mathbb{N}^+; k, l \in \mathbb{Z}$. Dann gilt:
\begin{align*}
    a \Mod m & = (a + k \cdot m) \Mod m \\
    (a \plusdot b) \Mod m &= ((a + k \cdot m) \plusdot (b + l \cdot m)) \Mod m
\end{align*}

\subsubsection*{Korollar zu Kongruenz 2}
(Ein Korollar ist eine Spezialisierung)

\begin{align*}
    a \Mod m &= (a \Mod m) \Mod m \\
    (a \plusdot b) \Mod m &= ((a \Mod m) \plusdot (b \Mod m)) \Mod m
\end{align*}

\subsection*{Operationen modulo m}
Sei $m \in \mathbb{N}^+$.
\begin{alignat*}{2}
    +_m, \cdot_m &: \mathbb{Z}_m \times \mathbb{Z}_m \xrightarrow{total} \mathbb{Z}_m & \\
    a +_m b &= (a + b) \Mod m \hfill &\text{Addition modulo m} \\
    a \cdot_m b &= (a \cdot b) \Mod m \hfill &\text{Multiplikation modulo m} 
\end{alignat*}


\section*{Homomorphismus}
Seien $a, b \in \mathbb{Z}, m \in \mathbb{N}^+$.
\begin{equation*}
    (a \plusdot b) \Mod m = (a \Mod m) \plusdotm (b \Mod m)
\end{equation*}

Sei $m \in \mathbb{N}^+$.
\begin{alignat*}{2}
    &(\mathbb{Z}_m, +_m, 0, -_m\bullet) & \hfill \text{abelsche Gruppe} \\
    &(\mathbb{Z}_m, \cdot_m, 1) & \hfill \text{abelsches Monoid (f"ur $m \geq 2$)} \\
\end{alignat*}

Sei $m \in \mathbb{N}^+, a \in \mathbb{Z}_m$. Dann:
\begin{equation*}
    -_m a = (-a) \Mod m = 
    \begin{cases*}
        m - a & f"ur $a \neq 0$ \\
        0     & f"ur $a = 0$
    \end{cases*}
\end{equation*}

\begin{align*}
    a \cdot_m b &= (a \cdot b) \Mod m \\
                &= a \cdot b - m \cdot q \\
                &= a \cdot b + m \cdot (-q)
\end{align*}


Sei $m \in \mathbb{N}^+$.
\begin{equation*}
    \mathbb{Z}^*_m := \{ a \in \mathbb{Z}_m | a \perp m \}
\end{equation*}
heisst \emph{reduzierte Menge der Reste modulo m}.

Sei $m \in \mathbb{N}^+$ mit $m \geq 2$. Dann ist
\begin{equation*}
    (\mathbb{Z}^*_m, \cdot_m, 1, \bullet^{-1}) 
\end{equation*}
eine abelsche Gruppe.

\begin{algorithmic}
    \Function{MultInvMod}{m, a : $\mathbb{Z}$}
        \State \Comment{requires $m \geq 2 \wedge 0 \leq a < m$}
        \State \Comment{ensures $ok \iff \gcd(m, a) = 1$}
        \State \Comment{ensures $ok \implies 1 \leq aInv < m \wedge a \cdot_m aInv = 1$}
        \State g, v : $\mathbb{Z}$
        \State (g, \_, v, \_, \_, \_) := EuclidExtNat(m, a)
        \State \Comment{$u \cdot m + v \cdot a = g$}
        \State ok := g = 1
        \State \If{ok}
            \State aInv := v mod m
        \EndIf
    \EndFunction
\end{algorithmic}

\section*{Abgeschlossenheit}
Definition: $M$ Menge, $\circ$ bin"are Operation auf $M$. Sei $N \subseteq M$.
$N$ heisst \emph{abgeschlossen unter $\circ$} wenn
\begin{equation*}
    (\forall a, b \in N | a \circ b \in N)
\end{equation*}

Satz: $m \in \mathbb{N}^+$. $\mathbb{Z}^*_m$ abgeschlossen unter $\cdot_m$

\section*{Eulersche $\varphi$-Funktion}
Definition:  Die Funktion
\begin{align*}
    \varphi &: \mathbb{N}^+ \xrightarrow{total} \mathbb{N}^+ \\
    \varphi(m) &= | \mathbb{Z}^*_m|
\end{align*}
heisst Eulersche $\varphi$-Funktion.

Satz: Sei $p, q$ prim, $p \neq q$, $k \in \mathbb{N}^+$.
\begin{align*}
    \varphi(p) &= p - 1 \\
    \varphi(p \cdot q) &= (p - 1) \cdot (q - 1) \\
    \varphi(p^k) &= p^{k - 1} \cdot (p - 1)
\end{align*}

\section*{Potenzen/Vielfache}
Definition: Sei $(M, \cdot, 1, \bullet^{-1})$ eine Gruppe (multiplikative Notation), $a \in M, n \in \mathbb{N}^+$.
\begin{align*}
    a^n &:= \underbrace{a \cdot a \cdot \dotsc \cdot a}_{\text{$n$ mal $a$, $n - 1$ mal $\cdot$}} \\
    a^0 &:= 1 \\
    a^{-n} &:= \left(a^{-1}\right)^n
\end{align*}

Sei $(M, +, 0, -\bullet)$ eine Gruppe (additive Noation),\\ 
$a \in M, n \in \mathbb{N}^+$.

\begin{align*}
    n \times a &:= \underbrace{a + a + \dotsc + a}_{n \text{ mal das } a} \\
    0 \times a &:= 0 \\
    (-n) \times a &:= n \times (-a)
\end{align*}

Satz: $(M, \cdot, 1, \bullet^{-1})$ Gruppe. $a \in M; m, n \in \mathbb{Z}$.
\begin{align*}
    a^m \cdot a^n &= a^{m+n} \\
    \left(a^m\right)^n &= a^{m \cdot n} 
\end{align*}

\section*{Modulare Exponentiation}
\begin{algorithmic}
    \Function{modular\_pow}{q, e, m}
        \State p := 1
        \State q := q mod m
        \While{$e > 0$}
            \If{e mod 2 == 1}
                \State p := (p * q) mod m
            \EndIf
            \State e := e / 2
            \State q := (q * q) mod m
        \EndWhile
        \State \Return p
    \EndFunction
\end{algorithmic}

\subsection*{Beispiel}

$(M, \circ, 1) := (\mathbb{Z}_{100}, \cdot_{100}, 1);\ 17^{20} = ?$

\vspace{1ex}

\begin{tabularx}{\linewidth}{YYY}
    \hline
     e & q & p \\
    \hline
    20 & 17 & 1 \\
    10 & $17 \cdot_{100} 17 = 89$ & 1 \\
    5 & $89 \cdot_{100} 89 = 21$ & 1 \\
    4 & 21 & $21 \cdot_{100} 1 = 21$ \\
    2 & $21 \cdot_{100} 21 = 41$ & 21 \\
    1 & $41 \cdot_{100} 41 = 81$ & 21 \\
    0 & 81 & $81 \cdot_{100} 21 = 1$ \\
    \hline
\end{tabularx}

\end{multicols*}

\end{document}


